\documentclass[12pt,a4paper]{article}

% Packages Ecriture
\usepackage[utf8]{inputenc} % Caractères accentués
\usepackage[french]{babel} % Traduction française
\usepackage[T1]{fontenc} % Codage des caractères
\usepackage{lmodern} % Polices vectorielles
\usepackage{dashundergaps}

% Packages Mise en page
\usepackage[margin=2cm]{geometry} % Marges
\usepackage{sectsty} % Définition des sections
\usepackage{titletoc} % Gestion des sections, titres ...
\usepackage{titlesec}
\usepackage{fancyhdr} %gestion des ET et PDP
%\usepackage{lscape}

% Packages mathématiques
\usepackage{amsmath} % Equations
\usepackage{amssymb} % Symboles
\usepackage{amsfonts} % Symbloes
\usepackage{akkadian}
\usepackage{bm}

% Packages Tableaux
\usepackage{tabularx} % Tableaux
\usepackage{multirow} %Gestion des lignes
\usepackage{multicol} % Gestion des colonnes
\usepackage{array} %Tableaux maths
\usepackage{arydshln} % Lignes en pointillés
\usepackage{fancybox} % Boites
\usepackage{nccrules} % pointillés

% Packages Figures et graphiques
\usepackage{graphics} % Inclusion de figures
\usepackage{graphicx} % Inclusion de figures
\usepackage{pstricks,pstricks-add,pst-plot,pst-eucl} % Graphiques
\usepackage[pstricks]{bclogo} % Logo
\usepackage{epsfig} % Inclusion de figures eps
           
% Tableaux
\newcolumntype{C}[1]{>{\centering\arraybackslash}p{#1cm}}
\renewcommand{\arraystretch}{1.5}
\setlength{\parindent}{0pt}
 
\newcommand{\babun}{{\AKK\dish}}
\newcommand{\babdeux}{{\AKK\min}}
\newcommand{\babtrois}{{\AKK\eshV}}
\newcommand{\babquatre}{{\AKK\shaII}}
\newcommand{\babcinq}{{\AKK\iaII}}
\newcommand{\babsix}{{\AKK\ashIII}}
\newcommand{\babsept}{{\AKK\imin}}
\newcommand{\babhuit}{{\AKK\ussu}}
\newcommand{\babneuf}{{\AKK\geshilimmu}}
\newcommand{\babdix}{{\AKK\AKKu}}
\newcommand{\babvingt}{{\AKK\man}}
\newcommand{\babtrente}{{\AKK\esh}}
\newcommand{\babquarante}{{\AKK\ninV}}
\newcommand{\babcinquante}{{\AKK\ninnu}}

\graphicspath{{../Images/}}

\begin{document}
\thispagestyle{empty}

%%%%%%%%%%%%%%%%%%%%%%%%%%%%%%%%%%

\begin{center}
   \bf LA NUMÉRATION DE LA CIVILISATION BABYLONIENNE \\ [1cm]
\end{center}
   
\noindent La civilisation Babylonienne à vécu en Mésopotamie environ entre $-3300$ av. J.-C. et $-539$ av. J.-C., c'est donc la troisième civilisation que nous étudions dans l'antiquité.

\begin{center}
   \includegraphics[width=13cm]{frise_historique}
\end{center}

{\bf\blue 1)} La {\bf Mésopotamie} est un terme d’origine grecque qui signifie littéralement \og entre les fleuves \fg. C’est un territoire délimité par le Tigre et l’Euphrate et qui est situé au cœur de ce que l’on appelle le croissant fertile. \\
Ci-dessous, la carte de gauche localise la Mésopotamie dans la période antique (carte 1), celle de droite est une carte politique actuelle du Moyen-Orient (carte 2).
\begin{center}
   \includegraphics[width=8.5cm]{../Images/Mesopotamie}\includegraphics[width=8.5cm]{../Images/Moyen_orient}
\end{center}
Quel est le principal pays actuel sur lequel s'étendait la Mésopotamie ? \\ [3mm]
\mbox{}\dashrulefill{4 3}{0.3}  \\

{\bf\blue 2)} L’{\bf écriture} est inventée en Mésopotamie vers 3300 avant J.-C., elle est le résultat de l’apparition des cités-états et de leur développement. En effet, les marchands, les artisans mais aussi les bergers et les paysans éprouvent le besoin de garder une trace des différents échanges commerciaux qu’ils effectuent. Au départ, les hommes mémorisent les différents inventaires à l’aide de petites billes d’argile, de taille et de formes diverses, les {\it calculi} (caillou en Latin).

\begin{center}
   \includegraphics[width=10cm]{Calculi}
\end{center}

\begin{minipage}{9cm}
   À votre avis, que représente le nombre ci-après ? \\ [3mm]
   \mbox{}\dashrulefill{4 3}{0.3}  \\ [3mm]
   \mbox{}\dashrulefill{4 3}{0.3}  \\
\end{minipage}
\;
\begin{minipage}{7.5cm}
   \includegraphics[width=7.5cm]{Calculi_bis}
\end{minipage}

\pagebreak
\thispagestyle{empty}

{\bf\blue 3)} Ce système évolue : peu à peu, les  calculi sont délaissés au profit de tablettes d’argile. Elles accueillent au départ des chiffres, puis une écriture constituée de {\bf pictogrammes} qui représentent les animaux ou les objets à comptabiliser. Les textes à écrire devenant de plus en plus longs, les pictogrammes sont simplifiés, c'est l’apparition de l'{\bf écriture cunéiforme}. \\
Ces signes : les coins \babdix et les clous \babun, sont imprimés par les {\bf scribes} sur une tablette d’argile humide à l’aide d’un {\bf calame} à bout triangulaire. \\
À l'aide des mots écrits en gras, compléter le nom des photos ci-dessous.
\begin{center}
   \includegraphics[height=3cm]{cuneiforme} \qquad \includegraphics[height=2cm]{calame} \qquad \includegraphics[height=3cm]{pictogrammes} \qquad \includegraphics[height=4.5cm]{scribe_m}
\end{center}
   \mbox{}\dashrulefill{4 3}{0.3} \quad \dashrulefill{4 3}{0.3} \quad \dashrulefill{4 3}{0.3} \quad \dashrulefill{4 3}{0.3} \\

{\bf\blue 4)} Les mathématiques cunéiformes reposent sur un système {\bf sexagésimal}, c’est-à-dire en base 60, qui s’appuie sur le système de clous et de chevrons (coins). Voilà comment on écrit les 15 premières nombres entiers ainsi que les dizaines. Compléter le tableau jusqu'à 59.

\begin{center}
\begin{tabular}{|*{10}{C{1.2}|}}
   \hline
   & \babun
   & \babdeux
   & \babtrois
   & \babquatre
   & \babcinq
   & \babsix
   & \babsept
   & \babhuit
   & \babneuf
   \\
   \hline
   \babdix
   & \babdix\!\!\!\babun
   & \babdix\!\!\!\babdeux
   & \babdix\!\!\!\babtrois
   & \babdix\!\!\!\babquatre
   & \babdix\!\!\!\babcinq
   & & & & \\
   \hline
   \babvingt & & & & & & & & & \\
   \hline
   \babtrente & & & & & & & & & \\
   \hline
   \babquarante & & & & & & & & & \\
   \hline
   \babcinquante & & & & & & & & & \\
   \hline
\end{tabular}
\end{center}

{\bf\blue 5)} À partir  de 60, la  numération devient {\bf positionnelle} : la valeur d'un signe dépend de sa position par rapport aux autres. On a par exemple :
\begin{center}
   \fbox{\babun\babtrois représente $1\times \bm{60}+3\times \bm{1}= 63$} \qquad \fbox{\babtrois\babvingt\!\!\!\babcinq représente $3\times\bm{60}+25\times\bm{1}= 205$}
\end{center}
Suivant ce modèle, transformer ces nombres en écriture cunéiforme en chiffres indo-arabe. \\ [3mm]
\babun\babneuf \mbox{}\dashrulefill{4 3}{0.3} \\ [3mm]
\babhuit\babdix\!\!\!\babquatre \mbox{}\dashrulefill{4 3}{0.3} \\ [3mm]
\babdix\!\!\!\babsept\babtrente\!\!\!\babdeux \mbox{}\dashrulefill{4 3}{0.3} \\ [3mm]

{\bf\blue 6)} Inversement, déterminer l'écriture cunéiforme des nombres suivants : \\ [3mm]
 78 : \dashrulefill{4 3}{0.3}  \\ [3mm]
462 : \dashrulefill{4 3}{0.3}  \\ [3mm]
2\,019 : \dashrulefill{4 3}{0.3} 
     
\pagebreak
\thispagestyle{empty}

%%%%%%%%%%%%%%%%%% Correction %%%%%%%%%%%%%%%%

%%%%%%%%%%%%%%%%%%%%%%%%%%%%%%%%%%

\begin{center}
   \bf LA NUMÉRATION DE LA CIVILISATION BABYLONIENNE \\ [1cm]
\end{center}
   
\noindent La civilisation Babylonienne à vécu en Mésopotamie environ entre $-3300$ av. J.-C. et $-539$ av. J.-C., c'est donc la troisième civilisation que nous étudions dans l'antiquité.

\begin{center}
   \includegraphics[width=13cm]{frise_historique}
\end{center}

{\bf\blue 1)} La {\bf Mésopotamie} est un terme d’origine grecque qui signifie littéralement \og entre les fleuves \fg. C’est un territoire délimité par le Tigre et l’Euphrate et qui est situé au cœur de ce que l’on appelle le croissant fertile. \\
Ci-dessous, la carte de gauche localise la Mésopotamie dans la période antique (carte 1), celle de droite est une carte politique actuelle du Moyen-Orient (carte 2).
\begin{center}
   \includegraphics[width=8.5cm]{../Images/Mesopotamie}\includegraphics[width=8.5cm]{../Images/Moyen_orient}
\end{center}
Quel est le principal pays actuel sur lequel s'étendait la Mésopotamie ? \\ [3mm]
{\red La Mésopotamie s'étendait essentiellement sur l'Irak (et un peu en Syrie).} \\

{\bf\blue 2)} L’{\bf écriture} est inventée en Mésopotamie vers 3300 avant J.-C., elle est le résultat de l’apparition des cités-états et de leur développement. En effet, les marchands, les artisans mais aussi les bergers et les paysans éprouvent le besoin de garder une trace des différents échanges commerciaux qu’ils effectuent. Au départ, les hommes mémorisent les différents inventaires à l’aide de petites billes d’argile, de taille et de formes diverses, les {\it calculi} (caillou en Latin).

\begin{center}
   \includegraphics[width=10cm]{Calculi}
\end{center}

\begin{minipage}{9cm}
   À votre avis, que représente le nombre ci-après ? \\ [3mm]
   {\red Il suffit d'additionner les valeurs des calculi : \\ [3mm]
   $36\,000+2\times3\,600+600+60+1 =43\,861$.}  \\ [3mm]
\end{minipage}
\;
\begin{minipage}{7.5cm}
   \includegraphics[width=7.5cm]{Calculi_bis}
\end{minipage}

\pagebreak
\thispagestyle{empty}

{\bf\blue 3)} Ce système évolue : peu à peu, les  calculi sont délaissés au profit de tablettes d’argile. Elles accueillent au départ des chiffres, puis une écriture constituée de {\bf pictogrammes} qui représentent les animaux ou les objets à comptabiliser. Les textes à écrire devenant de plus en plus longs, les pictogrammes sont simplifiés, c'est l’apparition de l'{\bf écriture cunéiforme}. \\
Ces signes : les coins \babdix et les clous \babun, sont imprimés par les {\bf scribes} sur une tablette d’argile humide à l’aide d’un {\bf calame} à bout triangulaire. \\
À l'aide des mots écrits en gras, compléter le nom des photos ci-dessous.
\begin{center}
   \includegraphics[height=3cm]{cuneiforme} \qquad \includegraphics[height=2cm]{calame} \qquad \includegraphics[height=3cm]{pictogrammes} \qquad \includegraphics[height=4.5cm]{scribe_m}
\end{center}
   {\red \quad écriture cunéiforme \hfill calame \qquad \hfill pictogrammes \hfill \quad scribe \hfill} \\

{\bf\blue 4)} Les mathématiques cunéiformes reposent sur un système {\bf sexagésimal}, c’est-à-dire en base 60, qui s’appuie sur le système de clous et de chevrons (coins). Voilà comment on écrit les 15 premières nombres entiers ainsi que les dizaines. Compléter le tableau jusqu'à 59.

\begin{center}
\begin{tabular}{|*{10}{C{1.2}|}}
   \hline
   & \babun
   & \babdeux
   & \babtrois
   & \babquatre
   & \babcinq
   & \babsix
   & \babsept
   & \babhuit
   & \babneuf
   \\
   \hline
   \babdix
   & \babdix\!\!\!\babun
   & \babdix\!\!\!\babdeux
   & \babdix\!\!\!\babtrois
   & \babdix\!\!\!\babquatre
   & \babdix\!\!\!\babcinq
   & \mbox{}{\red \babdix\!\!\!\babsix}
   & \mbox{}{\red \babdix\!\!\!\babsept}
   & \mbox{}{\red \babdix\!\!\!\babhuit}
   & \mbox{}{\red \babdix\!\!\!\babneuf} \\
   \hline
   \babvingt 
   & \mbox{}{\red \babvingt\!\!\!\babun}
   & \mbox{}{\red \babvingt\!\!\!\babdeux}
   & \mbox{}{\red \babvingt\!\!\!\babtrois}
   & \mbox{}{\red \babvingt\!\!\!\babquatre}
   & \mbox{}{\red \babvingt\!\!\!\babcinq}
   & \mbox{}{\red \babvingt\!\!\!\babsix}
   & \mbox{}{\red \babvingt\!\!\!\babsept}
   & \mbox{}{\red \babvingt\!\!\!\babhuit}
   & \mbox{}{\red \babvingt\!\!\!\babneuf} \\
   \hline
   \babtrente
   & \mbox{}{\red \babtrente\!\!\!\babun}
   & \mbox{}{\red \babtrente\!\!\!\babdeux}
   & \mbox{}{\red \babtrente\!\!\!\babtrois}
   & \mbox{}{\red \babtrente\!\!\!\babquatre}
   & \mbox{}{\red \babtrente\!\!\!\babcinq}
   & \mbox{}{\red \babtrente\!\!\!\babsix}
   & \mbox{}{\red \babtrente\!\!\!\babsept}
   & \mbox{}{\red \babtrente\!\!\!\babhuit}
   & \mbox{}{\red \babtrente\!\!\!\babneuf} \\
   \hline
   \babquarante
   & \mbox{}{\red \babquarante\!\!\!\babun}
   & \mbox{}{\red \babquarante\!\!\!\babdeux}
   & \mbox{}{\red \babquarante\!\!\!\babtrois}
   & \mbox{}{\red \babquarante\!\!\!\babquatre}
   & \mbox{}{\red \babquarante\!\!\!\babcinq}
   & \mbox{}{\red \babquarante\!\!\!\babsix}
   & \mbox{}{\red \babquarante\!\!\!\babsept}
   & \mbox{}{\red \babquarante\!\!\!\babhuit}
   & \mbox{}{\red \babquarante\!\!\!\babneuf} \\
   \hline
   \babcinquante
   & \mbox{}{\red \babcinquante\!\!\!\babun}
   & \mbox{}{\red \babcinquante\!\!\!\babdeux}
   & \mbox{}{\red \babcinquante\!\!\!\babtrois}
   & \mbox{}{\red \babcinquante\!\!\!\babquatre}
   & \mbox{}{\red \babcinquante\!\!\!\babcinq}
   & \mbox{}{\red \babcinquante\!\!\!\babsix}
   & \mbox{}{\red \babcinquante\!\!\!\babsept}
   & \mbox{}{\red \babcinquante\!\!\!\babhuit}
   & \mbox{}{\red \babcinquante\!\!\!\babneuf} \\
   \hline
\end{tabular}
\end{center}

{\bf\blue 5)} À partir  de 60, la  numération devient {\bf positionnelle} : la valeur d'un signe dépend de sa position par rapport aux autres. On a par exemple :
\begin{center}
   \fbox{\babun\babtrois représente $1\times \bm{60}+3\times \bm{1}= 63$} \qquad \fbox{\babtrois\babvingt\!\!\!\babcinq représente $3\times\bm{60}+25\times\bm{1}= 205$}
\end{center}
Suivant ce modèle, transformer ces nombres en écriture cunéiforme en chiffres indo-arabe. \\ [3mm]
\babun\babneuf {\red \qquad représente $1\times\bm{60}+9\times\bm{1}= 69$}  \\ [3mm]
\babhuit\babdix\!\!\!\babquatre {\red \qquad représente $8\times\bm{60}+14\times\bm{1}= 494$} \\ [3mm]
\babdix\!\!\!\babsept\babtrente\!\!\!\babdeux {\red \qquad représente $17\times\bm{60}+32\times\bm{1}= 1\,052$} \\ [3mm]

{\bf\blue 6)} Inversement, déterminer l'écriture cunéiforme des nombres suivants : \\ [3mm]
 78 {\red $=1\times\bm{60}+18\times\bm{1}$, on obtient donc \babun\babdix\!\!\!\babhuit} \\ [3mm]
462 {\red $=7\times\bm{60}+42\times\bm{1}$, on obtient donc \babsept\babquarante\!\!\!\babdeux} \\ [3mm]
2\,019 {\red $=33\times\bm{60}+39\times\bm{1}$, on obtient donc \babtrente\!\!\!\babtrois\babtrente\!\!\!\babneuf} \\ [3mm]
     
     
\end{document}

