\documentclass[12pt,a4paper]{article}

% Packages Ecriture
\usepackage[utf8]{inputenc} % Caractères accentués
\usepackage[french]{babel} % Traduction française
\usepackage[T1]{fontenc} % Codage des caractères
\usepackage{lmodern} % Polices vectorielles
\usepackage{dashundergaps}

% Packages Mise en page
\usepackage[margin=2cm]{geometry} % Marges
\usepackage{sectsty} % Définition des sections
\usepackage{titletoc} % Gestion des sections, titres ...
\usepackage{titlesec}
\usepackage{fancyhdr} %gestion des ET et PDP
%\usepackage{lscape}

% Packages mathématiques
\usepackage{amsmath} % Equations
\usepackage{amssymb} % Symboles
\usepackage{amsfonts} % Symbloes
\usepackage{hieroglf}
\newcommand{\cRm}[1]{\textsc{\romannumeral #1}}

% Packages Tableaux
\usepackage{tabularx} % Tableaux
\usepackage{multirow} %Gestion des lignes
\usepackage{multicol} % Gestion des colonnes
\usepackage{array} %Tableaux maths
\usepackage{arydshln} % Lignes en pointillés
\usepackage{fancybox} % Boites
\usepackage{nccrules} % pointillés

% Packages Figures et graphiques
\usepackage{graphics} % Inclusion de figures
\usepackage{graphicx} % Inclusion de figures
\usepackage{pstricks,pstricks-add,pst-plot,pst-eucl} % Graphiques
\usepackage[pstricks]{bclogo} % Logo
\usepackage{epsfig} % Inclusion de figures eps
           
% Tableaux
\newcolumntype{C}[1]{>{\centering\arraybackslash}p{#1cm}}
\renewcommand{\arraystretch}{1.3}
\setlength{\parindent}{0pt}
 
\graphicspath{{../Images/}}

\begin{document}
\thispagestyle{empty}

%%%%%%%%%%%%%%%%%%%%%%%%%%%%%%%%%%

\begin{center}
   \bf LA NUMÉRATION DE L'EMPIRE ROMAIN \\ [1cm]
\end{center}
   
\noindent L'empire romain est le nom de la période de la Rome antique, s'étendant entre 27 avant J.-C. et 476 après J.-C., donc dans l'antiquité.

\begin{center}
   \includegraphics[width=13cm]{frise_historique}
\end{center}

{\bf\blue 1)} Les Romains ont peu contribué au développement des sciences, mais il se sont approprié les connaissances des peuples qu'ils ont conquis. Il sont cependant de très grands architectes : pourriez-vous donner le nom de ces bâtiments et dire à quoi ils pouvaient servir ? \\
   
\begin{minipage}{6cm}
   \includegraphics[width=6cm]{../Images/amphitheatre}
\end{minipage}
\qquad
\begin{minipage}{10cm}
   Nom : \dashrulefill{4 3}{0.3} \\ [3mm]
   Fonction : \dashrulefill{4 3}{0.3} \\ [3mm]
   \mbox{}\dashrulefill{4 3}{0.3}  \\ [3mm]
\end{minipage}

\bigskip

\begin{minipage}{11cm}
   Nom : \dashrulefill{4 3}{0.3} \\ [3mm]
   Fonction : \dashrulefill{4 3}{0.3} \\ [3mm]
   \mbox{}\dashrulefill{4 3}{0.3} \\ [3mm]
\end{minipage}
\qquad
\begin{minipage}{5cm}
   \includegraphics[width=5cm]{../Images/thermes}
\end{minipage}

\bigskip

\begin{minipage}{4.5cm}
   \includegraphics[width=4.5cm]{../Images/temple}
\end{minipage}
\qquad
\begin{minipage}{11.5cm}
   Nom : \dashrulefill{4 3}{0.3} \\ [3mm]
   Fonction : \dashrulefill{4 3}{0.3} \\ [3mm]
   \mbox{}\dashrulefill{4 3}{0.3} \\ [3mm]
\end{minipage}

\bigskip

\begin{minipage}{10cm}
   Nom : \dashrulefill{4 3}{0.3} \\ [3mm]
   Fonction : \dashrulefill{4 3}{0.3} \\ [3mm]
   \mbox{}\dashrulefill{4 3}{0.3} \\ [3mm]
\end{minipage}
\qquad
\begin{minipage}{6cm}
   \includegraphics[width=6cm]{../Images/port}
\end{minipage}

\bigskip

\begin{minipage}{5cm}
   \includegraphics[width=5cm]{../Images/aqueduc}
\end{minipage}
\qquad
\begin{minipage}{11cm}
   Nom : \dashrulefill{4 3}{0.3} \\ [3mm]
   Fonction : \dashrulefill{4 3}{0.3} \\ [3mm]
   \mbox{}\dashrulefill{4 3}{0.3} \\ [3mm]
\end{minipage}

\pagebreak
\thispagestyle{empty}
   
  {\bf\blue 2)} Pour bâtir, les Romains ont besoin de nombreuses connaissances en géométrie, en dessin, en mathématiques et en optique. Ils ont besoin de mesurer beaucoup de choses et utilisent les nombres romains. Ils nous ont laissé un système de numération qu'il nous arrive encore d'utiliser. La numération romaine comprend sept chiffres représentés par des lettres. Dans le tableau ci-dessous, on a représenté ces sept signes ainsi que leur nom en latin. Sauriez-vous retrouver la valeur de chaque chiffre ?
\begin{center} 
  \footnotesize
   \begin{tabular}{|*{7}{C{1.8}|}}
      \hline
      \cRm{1} & \cRm{5} & \cRm{10} & \cRm{50} & \cRm{100} & \cRm{500} & \cRm{1000} \\
      \hline
      unus & quinque & decem & quinquaginta & centum & quingenti & mille \\
      \hline
       & & & & & & \\
       \hline
      
   \end{tabular}
\end{center}
   
   {\bf\blue 3)} Ce système utilise le groupement par dix et un groupement auxiliaires par cinq. Il s'agit d'un système \textbf{additif} et \textbf{soustractif} qui dépend de l'ordre des chiffres. On a par exemple : \\
   \begin{tabular}{p{0.8cm}cp{2.7cm}cp{3.9cm}cp{6cm}}
      \cRm{11} & $\Rightarrow$ & 10 ; 1 & $\Rightarrow$ & $10>1$ & $\Rightarrow$ & $10+1 =11$ \\
      \cRm{9} & $\Rightarrow$ & 1 ; 10 & $\Rightarrow$ & $\underline{1<10}$ & $\Rightarrow$ & $10\,\textcircled{$-$}\,1 =9$ \\
      \cRm{153} & $\Rightarrow$ & 100 ; 50 ; 1 ;1 ; 1 & $\Rightarrow$ & $100>50>1=1=1$ & $\Rightarrow$ & $100+50+1+1+1 =153$ \\
      \cRm{94} & $\Rightarrow$ & 10 ; 100 ; 1 ; 5 & $\Rightarrow$ & $\underline{10<100}>\underline{1<5}$ & $\Rightarrow$ & $100\,\textcircled{$-$}\,10+5\,\textcircled{$-$}\,1 =94$ \\
   \end{tabular} \\ [1mm]
   Après avoir observé ces exemples, conjecturer trois règles d'écriture des chiffres romains. \\ [4mm]
   $\bullet$\dashrulefill{4 3}{0.3} \\ [3mm]
   $\bullet$\dashrulefill{4 3}{0.3} \\ [3mm]
   $\bullet$\dashrulefill{4 3}{0.3} \\ [1mm]
   
   {\bf\blue 4)} Colorier les cases qui forment la suite de nombres de 30 à 40.
   \begin{center}
   \large
    \begin{tabular}{|*{7}{C{1.6}|}}
       \hline
       \cRm{30} & \cRm{31} & \cRm{20} & \cRm{64} & \cRm{41} & \cRm{39} & \cRm{40} \\
       \hline
       \cRm{28} & \cRm{32} & \cRm{39} & \cRm{36} & \cRm{37} & \cRm{38} & \cRm{40} \\
       \hline
       \cRm{24} & \cRm{33} & \cRm{34} & \cRm{35} & \cRm{18} & \cRm{27} & \cRm{140} \\
       \hline
    \end{tabular}
   \end{center}
   {\bf\blue 5)} Transcrire les nombres suivants en chiffres indo-arabes.
   \begin{multicols}{2}
      Chapitre \cRm{22} : \dashrulefill{4 3}{0.3} \\ [1mm]
      Le \cRm{15}\up{e} siècle : \dashrulefill{4 3}{0.3} \\ [1mm]
      Napoléon \cRm{3} : \dashrulefill{4 3}{0.3} \\ [1mm]
      Louis \cRm{18} : \dashrulefill{4 3}{0.3} \\ [1mm]
      Mohammed \cRm{6} : \dashrulefill{4 3}{0.3} \\ [1mm]
      La \cRm{8}\up{e} avenue : \dashrulefill{4 3}{0.3} \\ [1mm]
      \cRm{2019} : \dashrulefill{4 3}{0.3} \\ [1mm]
      Henri \cRm{4} : \dashrulefill{4 3}{0.3} \\ [1mm]
      Louis \cRm{14} : \dashrulefill{4 3}{0.3} \\ [1mm]
      Le \cRm{19}\up{e} siècle : \dashrulefill{4 3}{0.3} \\ [1mm]
      La \cRm{5}\up{e} République : \dashrulefill{4 3}{0.3} \\ [1mm]
      Star Wars \cRm{7} : \dashrulefill{4 3}{0.3} \\ [1mm]
      Final Fantasy \cRm{15} : \dashrulefill{4 3}{0.3} \\ [1mm]
      \cRm{999} : \dashrulefill{4 3}{0.3}
   \end{multicols}
   
   {\bf\blue 6)} Le système est vite limité pour écrire des grands nombres (supérieurs à 4\,999). Plus tard, les romains ajouteront une barre au dessus des signes multipliant par 1\,000 leur valeur initiale. \\
   Par exemple, $\overline{\cRm{10}}\overline{\cRm1}\cRm{253}$ est égal à $10\times1\,000+1\times1\,000+100+100+50+1+1+1 =11\,253$. \\
   Écrire les nombre suivants en chiffres romains. \\ [3mm]
   1\,000\,064 : \dashrulefill{4 3}{0.3} 357\,001 : \dashrulefill{4 3}{0.3}
   
\pagebreak
\thispagestyle{empty}

\begin{center}
   \bf LA NUMÉRATION DE L'EMPIRE ROMAIN \\ [1cm]
\end{center}
   
\noindent L'empire romain est le nom de la période de la Rome antique, s'étendant entre 27 avant J.-C. et 476 après J.-C., donc dans l'antiquité.

\begin{center}
   \includegraphics[width=13cm]{frise_historique}
\end{center}

{\bf\blue 1)} Les Romains ont peu contribué au développement des sciences, mais il se sont approprié les connaissances des peuples qu'ils ont conquis. Il sont cependant de très grands architectes : pourriez-vous donner le nom de ces bâtiments romains et dire à quoi ils pouvaient servir ? \\
   
\begin{minipage}{6cm}
   \includegraphics[width=6cm]{../Images/amphitheatre}
\end{minipage}
\qquad
\begin{minipage}{10cm}
   Nom : {\red amphithéâtre.} \\ [3mm]
   Fonction : {\red édifice de forme elliptique à gradins organisé autour d'une arène où étaient donnés des spectacles de gladiateurs.}
\end{minipage}

\bigskip

\begin{minipage}{11cm}
   Nom : {\red thermes.} \\ [3mm]
   Fonction : {\red établissement abritant des bains privés ou publics permettant aux populations de se baigner, et parfois de se laver (du grec {\it thermos}, chaud).}
\end{minipage}
\qquad
\begin{minipage}{5cm}
   \includegraphics[width=5cm]{../Images/thermes}
\end{minipage}

\bigskip

\begin{minipage}{4.5cm}
   \includegraphics[width=4.5cm]{../Images/temple}
\end{minipage}
\qquad
\begin{minipage}{11.5cm}
   Nom : {\red temple.} \\ [3mm]
   Fonction : {\red édifice dédié à la pratique extérieure du culte des romains.}
\end{minipage}

\bigskip

\begin{minipage}{10cm}
   Nom : {\red port.} \\ [3mm]
   Fonction : {\red structure permettant d'accueillir les ravitaillements essentiellement pour le commerce et à abriter les flottes romaines.}
\end{minipage}
\qquad
\begin{minipage}{6cm}
   \includegraphics[width=6cm]{../Images/port}
\end{minipage}

\bigskip

\begin{minipage}{5cm}
   \includegraphics[width=5cm]{../Images/aqueduc}
\end{minipage}
\qquad
\begin{minipage}{11cm}
   Nom : {\red aqueduc.} \\ [3mm]
   Fonction : {\red canalisation destinée à conduire les eaux pour la consommation des villes (du latin {\it aquae}, eau et {\it ductus} conduire).}
\end{minipage}

\pagebreak
\thispagestyle{empty}
   
  {\bf\blue 2)} Pour bâtir, les Romains ont besoin de nombreuses connaissances en géométrie, en dessin, en mathématiques et en optique. Ils ont besoin de mesurer beaucoup de choses et utilisent les nombres romains. Ils nous ont laissé un système de numération qu'il nous arrive encore d'utiliser. La numération romaine comprend sept chiffres représentés par des lettres. Dans le tableau ci-dessous, on a représenté ces sept signes ainsi que leur nom en latin. Sauriez-vous retrouver la valeur de chaque chiffre ?
\begin{center} 
  \footnotesize
   \begin{tabular}{|*{7}{C{1.8}|}}
      \hline
      \cRm{1} & \cRm{5} & \cRm{10} & \cRm{50} & \cRm{100} & \cRm{500} & \cRm{1000} \\
      \hline
      unus & quinque & decem & quinquaginta & centum & quingenti & mille \\
      \hline
      \mbox{\red 1} & \mbox{\red 5} & \mbox{\red 10} & \mbox{\red 50} & \mbox{\red 100} & \mbox{\red 500} & \mbox{\red 1\,000} \\
       \hline
      
   \end{tabular}
\end{center}
   
   {\bf\blue 3)} Ce système utilise le groupement par dix et un groupement auxiliaires par cinq. Il s'agit d'un système \textbf{additif} et \textbf{soustractif} qui dépend de l'ordre des chiffres. On a par exemple : \\
   \begin{tabular}{p{0.8cm}cp{2.7cm}cp{3.9cm}cp{6cm}}
      \cRm{11} & $\Rightarrow$ & 10 ; 1 & $\Rightarrow$ & $10>1$ & $\Rightarrow$ & $10+1 =11$ \\
      \cRm{9} & $\Rightarrow$ & 1 ; 10 & $\Rightarrow$ & $\underline{1<10}$ & $\Rightarrow$ & $10\,\textcircled{$-$}\,1 =9$ \\
      \cRm{153} & $\Rightarrow$ & 100 ; 50 ; 1 ;1 ; 1 & $\Rightarrow$ & $100>50>1=1=1$ & $\Rightarrow$ & $100+50+1+1+1 =153$ \\
      \cRm{94} & $\Rightarrow$ & 10 ; 100 ; 1 ; 5 & $\Rightarrow$ & $\underline{10<100}>\underline{1<5}$ & $\Rightarrow$ & $100\,\textcircled{$-$}\,10+5\,\textcircled{$-$}\,1 =94$ \\
   \end{tabular} \\ [1mm]
   Après avoir observé ces exemples, conjecturez trois règles d'écriture des chiffres romains. \\ [4mm]
   $\bullet$ {\red Principe additif : tout nombre placé à la droite d’un autre nombre qui lui est supérieur ou égal s’ajoute à celui-ci.} \\
   $\bullet$ {\red Principe soustractif : tout nombre placé à la gauche d’un autre nombre qui lui est supérieur doit être soustrait du nombre à sa droite.} \\
   $\bullet$ {\red La même lettre peut être employée trois fois consécutivement au maximum sauf pour \cRm{1000}.} \\
   $\bullet$ {\red Les valeurs sont groupées en ordre décroissant, sauf pour les valeurs à retrancher.} \\
   
   {\bf\blue 4)} Coloriez les cases qui forment la suite de nombres de 30 à 40.
   \begin{center}
   \large
    \begin{tabular}{|*{7}{C{1.6}|}}
       \hline
       \mbox{\red\cRm{30}} & \mbox{\red\cRm{31}} & \cRm{20} & \cRm{64} & \cRm{41} & \mbox{\red\cRm{39}} & \mbox{\red\cRm{40}} \\
       \hline
       \cRm{28} & \mbox{\red\cRm{32}} & \cRm{29} & \mbox{\red\cRm{36}} & \mbox{\red\cRm{37}} & \mbox{\red\cRm{38}} & \cRm{60} \\
       \hline
       \cRm{24} & \mbox{\red\cRm{33}} & \mbox{\red\cRm{34}} & \mbox{\red\cRm{35}} & \cRm{18} & \cRm{27} & \cRm{140} \\
       \hline
    \end{tabular}
   \end{center}
   {\bf\blue 5)} Écrivez les nombres suivants en chiffres indo-arabes.
   \begin{multicols}{2}
      Chapitre \cRm{22} : {\red 22} \\ [1mm]
      Le \cRm{15}\up{e} siècle : {\red 15} \\ [1mm]
      Napoléon \cRm{3} : {\red 3} \\ [1mm]
      Louis \cRm{18} : {\red 18} \\ [1mm]
      Mohammed \cRm{6} : {\red 6} \\ [1mm]
      La \cRm{8}\up{e} avenue : {\red 8} \\ [1mm]
      \cRm{2019} : {\red 2019} \\ [1mm]
      Henri \cRm{4} : {\red 4} \\ [1mm]
      Louis \cRm{14} : {\red 14} \\ [1mm]
      Le \cRm{19}\up{e} siècle : {\red 19} \\ [1mm]
      La \cRm{5}\up{e} République : {\red 5} \\ [1mm]
      Star Wars \cRm{7} : {\red 7} \\ [1mm]
      Final Fantasy \cRm{15} : {\red 15} \\ [1mm]
      \cRm{999} : {\red 999}
   \end{multicols}
   
   {\bf\blue 6)} Le système est vite limité pour écrire des grands nombres (supérieurs à 4\,999). Plus tard, les romains ajouteront une barre au dessus des signes multipliant par 1\,000 leur valeur initiale. \\
   Par exemple, $\overline{\cRm{11}}\cRm{253}$ est égal à $10\times1\,000+1\times1\,000+100+100+50+1+1+1 =11\,253$. \\
   Écrire les nombre suivants en chiffres romains. \\ [3mm]
   1\,000\,064 : {\red $\overline{\cRm{1000}}$\cRm{64}} \hspace*{5.2cm} 357\,001: {\red $\overline{\cRm{357}}$\cRm{1}}
   
\end{document}

