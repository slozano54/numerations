\documentclass[french,12pt]{article}

% ============================================================================================
% ======= Configurations 
% ============================================================================================

% ======= Ajout de paquets
% ============================================================================================
% ======= Paquets
% ============================================================================================
\RequirePackage{fancyhdr}
\RequirePackage{fancyvrb}

% ============================================================================================
% ======= Pour le paquet ProfCollege
% ============================================================================================
\usepackage{ProfCollege}

% Pour la gestion des fontes maths notamment en mode LuaLaTeX.
\usepackage{unicode-math}
% Par exemple, une fonte sans serif pour les briques Scratch.
\newfontfamily\myfontScratch[]{FreeSans}

% ============================================================================================
% ======= Le reste
% ============================================================================================

% Pour générer du faux texte
\usepackage{lipsum}

% Pour les hyperliens, notamment des sommaires cliquables
\usepackage{hyperref}


% ======= Ajout de commandes
% ============================================================================================
% ======= 1ere de couverture
% ============================================================================================

\newcommand\myAuthorName{Mireille GAIN et Sébastien LOZANO}
\newcommand\myAuthorSchoolName{Établissements}
\newcounter{postCurrentSchoolYear}
\setcounter{postCurrentSchoolYear}{\the\year}
\addtocounter{postCurrentSchoolYear}{1}
\newcommand\currentSchoolYear{Année  \the\year ~- \thepostCurrentSchoolYear}

% ============================================================================================
% ======= Sommaire
% ============================================================================================
\renewcommand{\contentsname}{Sommaire}           % Changer le nom de la table des matières

% ============================================================================================
% ======= Styles
% ============================================================================================

\fancypagestyle{firstCover}{%   1ere de couverture
    \fancyhf{}%                 On initialise headers and footers à rien du tout !        
    \renewcommand{\headrulewidth}{0pt}%trait horizontal pour l'en-tête
    %\renewcommand{\footrulewidth}{0.4pt}%trait horizontal pour le pied de page    
}
\fancypagestyle{backCover}{%   4ere de couverture
    \fancyhf{}%                 On initialise headers and footers à rien du tout !        
    %\renewcommand{\headrulewidth}{0pt}%trait horizontal pour l'en-tête
    %\renewcommand{\footrulewidth}{0.4pt}%trait horizontal pour le pied de page
}

% ============================================================================================
% Factorisation de commandes
% ============================================================================================


% ======= Pour inclure de façon sélective
\includeonly{%
	./numerationEgyptienne/groupe1.tex,
	./numerationEgyptienne/groupe2.tex,
	./numerationRomaine/groupe1.tex,
	./numerationRomaine/groupe2.tex,
	./numerationShadok/groupe1.tex,
	./numerationShadok/groupe2.tex,
	./numerationInca/groupe1.tex,
	./numerationInca/groupe2.tex,
	./numerationChinoise/groupe1.tex,
	./numerationChinoise/groupe2.tex,
	./numerationBabylonienne/groupe1.tex,
	./numerationBabylonienne/groupe2.tex,
	./numerationBinaire/groupe1.tex,
	./numerationBinaire/groupe2.tex,
	./numerationTriozon/groupe1.tex,
	./numerationTriozon/groupe2.tex,
	./numerationDni/groupe1.tex,
	./numerationDni/groupe2.tex,
}
% ============================================================================================
% ======= Début du document
% ============================================================================================
\begin{document}
    \input{./incFirstCover.tex}            				% On inclut la 1ere de couverture

    \tableofcontents									% Sommaire
	
% ======= On inclut la numération Egyptienne
	\def\currentpath{./numerationEgyptienne}            % On définit le répertoire courant
	\clearpage
	\section{Numération Inca}
Début commun du doc
\lipsum[1-2]	
	\subsection{Spécifique Groupe1}
	\subsection{Spécifique Groupe2}
	\input{\currentpath/finCommune.tex}

% ======= On inclut la numération Romaine
	\def\currentpath{./numerationRomaine}            	% On définit le répertoire courant
	\clearpage
	\section{Numération Inca}
Début commun du doc
\lipsum[1-2]	
	\subsection{Spécifique Groupe1}
	\subsection{Spécifique Groupe2}
	\input{\currentpath/finCommune.tex}

% ======= On inclut la numération Shadok
	\def\currentpath{./numerationShadok}            	% On définit le répertoire courant
	\clearpage
	\section{Numération Inca}
Début commun du doc
\lipsum[1-2]	
	\subsection{Spécifique Groupe1}
	\subsection{Spécifique Groupe2}
	\input{\currentpath/finCommune.tex}

% ======= On inclut la numération Inca
	\def\currentpath{./numerationInca}            		% On définit le répertoire courant
	\clearpage
	\section{Numération Inca}
Début commun du doc
\lipsum[1-2]	
	\subsection{Spécifique Groupe1}
	\subsection{Spécifique Groupe2}
	\input{\currentpath/finCommune.tex}

% ======= On inclut la numération Chinoise
	\def\currentpath{./numerationChinoise}            	% On définit le répertoire courant
	\clearpage
	\section{Numération Inca}
Début commun du doc
\lipsum[1-2]	
	\subsection{Spécifique Groupe1}
	\subsection{Spécifique Groupe2}
	\input{\currentpath/finCommune.tex}

% ======= On inclut la numération Babylonienne
	\def\currentpath{./numerationBabylonienne}           % On définit le répertoire courant
	\clearpage
	\section{Numération Inca}
Début commun du doc
\lipsum[1-2]	
	\subsection{Spécifique Groupe1}
	\subsection{Spécifique Groupe2}
	\input{\currentpath/finCommune.tex}

% ======= On inclut la numération Binaire
	\def\currentpath{./numerationBinaire}            	% On définit le répertoire courant
	\clearpage
	\section{Numération Inca}
Début commun du doc
\lipsum[1-2]	
	\subsection{Spécifique Groupe1}
	\subsection{Spécifique Groupe2}
	\input{\currentpath/finCommune.tex}

% ======= On inclut la numération Triozon
	\def\currentpath{./numerationTriozon}            	% On définit le répertoire courant
	\clearpage
	\section{Numération Inca}
Début commun du doc
\lipsum[1-2]	
	\subsection{Spécifique Groupe1}
	\subsection{Spécifique Groupe2}
	\input{\currentpath/finCommune.tex}

% ======= On inclut la numération D'ni
	\def\currentpath{./numerationDni}            		% On définit le répertoire courant
	\clearpage
	\section{Numération Inca}
Début commun du doc
\lipsum[1-2]	
	\subsection{Spécifique Groupe1}
	\subsection{Spécifique Groupe2}
	\input{\currentpath/finCommune.tex}

    \pagestyle{backCover}
\parindent=0pt
Texte \hspace*{\stretch{1}} TP numérations \LaTeX 

Texte \hspace*{\stretch{1}} Paquet ProfCollege
\vspace*{\stretch{1}}
\begin{center}
    \includegraphics[scale=0.5]{images/8.png}%
\end{center}
\vspace*{\stretch{1}}
\hrulefill

Du texte ici si je veux \ldots

\hrulefill
\vspace*{1cm}
\begin{center}\bfseries\Large
    \myAuthorName
\end{center}
\begin{center}\bfseries
    \myAuthorSchoolName

    \currentSchoolYear
\end{center}    
    
\vspace*{\stretch{2}}
\begin{flushright}
       Dernière mise à jour le \today 
\end{flushright}   
          				% On inclut la 4e de couverture
\end{document}